\begin{table}[]
\centering
% \caption{\small{Mathematical notations used for theoretical attribution of SCHYS.}}
\caption{\small{Description of mathematical notations for SCRR.}}
\vspace{-3mm}
\label{tab:notations}
\resizebox{0.6\linewidth}{!}{
\begin{tabular}{l|l}
\rowcolor[HTML]{DAE8FC} 
{\color[HTML]{000000} \small{Notation}} & {\color[HTML]{000000} \small{Description}} \\ \hline
  \small{$p_f^j$}  &   \small{Packet $j$ of flow $f$}       \\ \hline
    \small{$l(p_f^j)$}  &  \small{Length of packet $p_f^j$}  \\ \hline
  \small{$c(k)$}  &   \small{Global virtual clock at scheduling round $k$}       \\ \hline
   \small{$v(p_f^j)$}   &   \small{Virtual time of packet $(p_f^j)$}       \\ \hline
     \small{$r_f$}  &  \small{Rate (weight) of flow $f$}   \\ \hline
     \small{$N$} & \small{Number of packets in the scheduler} \\ \hline
     \small{$Q$} & \small{Vector of active sub-queues in the scheduler} \\ \hline
     \small{$n$} & \small{Number of active sub-queues in the scheduler} \\ \hline
     \small{$M$} & \small{Maximum length among all packets} 
\end{tabular}
}
\vspace{-2mm}
\end{table}
